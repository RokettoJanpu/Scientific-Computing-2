\documentclass{article}
\usepackage{amsfonts, amsthm, amsmath, amssymb, mathtools, ulem, mathrsfs, physics, esint, siunitx, tikz-cd}
\usepackage{pdfpages, fullpage, color, microtype, cancel, textcomp, markdown, hyperref, graphicx}
\usepackage{enumitem}
\usepackage{algorithm}
\usepackage{algpseudocode}
\graphicspath{{./images/}}
\usepackage[english]{babel}
\usepackage[autostyle, english=american]{csquotes}
\MakeOuterQuote{"}
\usepackage{xparse}
\usepackage{tikz}

\usepackage{calligra}
\DeclareMathAlphabet{\mathcalligra}{T1}{calligra}{m}{n}
\DeclareFontShape{T1}{calligra}{m}{n}{<->s*[2.2]callig15}{}
\newcommand{\script}[1]{\ensuremath{\mathcalligra{#1}}}
\newcommand{\scr}{\script r}

% fonts
\def\mbb#1{\mathbb{#1}}
\def\mfk#1{\mathfrak{#1}}
\def\mbf#1{\mathbf{#1}}
\def\tbf#1{\textbf{#1}}

% common bold letters
\def\bP{\mbb{P}}
\def\bC{\mbb{C}}
\def\bH{\mbb{H}}
\def\bI{\mbb{I}}
\def\bR{\mbb{R}}
\def\bQ{\mbb{Q}}
\def\bZ{\mbb{Z}}
\def\bN{\mbb{N}}

% brackets
\newcommand{\br}[1]{\left(#1\right)}
\newcommand{\sbr}[1]{\left[#1\right]}
\newcommand{\brc}[1]{\left\{#1\right\}}
\newcommand{\lbr}[1]{\left\langle#1\right\rangle}

% vectors
\renewcommand{\i}{\hat{\imath}}
\renewcommand{\j}{\hat{\jmath}}
\renewcommand{\k}{\hat{k}}
\newcommand{\proj}[2]{\text{proj}_{#2}\br{#1}}
\newcommand{\m}[2][b]{\begin{#1matrix}#2\end{#1matrix}}
\newcommand{\arr}[3][\sbr]{#1{\begin{array}{#2}#3\end{array}}}

% misc
\NewDocumentCommand{\seq}{O{n} O{1} O{\infty} m}{\br{#4}_{{#1}={#2}}^{#3}}
\NewDocumentCommand{\app}{O{x} O{\infty}}{\xrightarrow{#1\to#2}}
\newcommand{\sm}{\setminus}
\newcommand{\sse}{\subseteq}
\renewcommand{\ss}{\subset}
\newcommand{\vn}{\varnothing}
\newcommand{\lc}{\epsilon_{ijk}}
\newcommand{\ep}{\epsilon}
\newcommand{\vp}{\varphi}
\renewcommand{\th}{\theta}
\newcommand{\cjg}[1]{\overline{#1}}
\newcommand{\inv}{^{-1}}
\DeclareMathOperator{\im}{im}
\DeclareMathOperator{\id}{id}
\newcommand{\ans}{\tbf{Ans. }}
\newcommand{\pf}{\tbf{Pf. }}
\newcommand{\imp}{\implies}
\newcommand{\impleft}{\reflectbox{$\implies$}}
\newcommand{\ck}{\frac1{4\pi\ep_0}}
\newcommand{\ckb}{4\pi\ep_0}
\newcommand{\sto}{\longrightarrow}
\DeclareMathOperator{\cl}{cl}
\DeclareMathOperator{\intt}{int}
\DeclareMathOperator{\bd}{bd}
\DeclareMathOperator{\Span}{span}
\newcommand{\floor}[1]{\left\lfloor#1\right\rfloor}
\newcommand{\ceil}[1]{\left\lceil#1\right\rceil}
\newcommand{\fxn}[5]{#1:\begin{array}{rcl}#2&\longrightarrow & #3\\[-0.5mm]#4&\longmapsto &#5\end{array}}
\newcommand{\sep}[1][.5cm]{\vspace{#1}}
\DeclareMathOperator{\card}{card}
\renewcommand{\ip}[2]{\lbr{#1,#2}}
\renewcommand{\bar}{\overline}
\DeclareMathOperator{\cis}{cis}
\DeclareMathOperator{\Arg}{Arg}
\newcommand{\ptl}{\partial}

% title
\title{Scientific Computing HW 8}
\author{Ryan Chen}
%\date{\today}
\setlength{\parindent}{0pt}


\begin{document}
	
\maketitle



\tbf{Problem 2.}

\begin{enumerate}[label=(\alph*)]
	
\item \pf Write
$$u(x) = \int_0^1 G(x,y)f(y)dy
= \int_0^x G(x,y)f(y)dy + \int_x^1 G(x,y)f(y)dy$$
$$= (1-x)\int_0^x yf(y)dy + x\int_x^1 (1-y)f(y)dy
= (1-x)\int_0^x yf(y)dy - x\int_1^x (1-y)f(y)dy$$
Then compute
$$u'(x) = -\int_0^x yf(y)dy + (1-x)xf(x) - \int_1^x (1-y)f(y)dy - x(1-x)f(x)$$
$$= -\int_0^x yf(y)dy - \int_1^x f(y)dy + \int_1^x yf(y)dy
= -\int_1^x f(y)dy$$
$$\imp u''(x) = -f(x)$$


\item \pf Compute
$$G_y(x,y) =
\begin{cases}
	1-x, & y<x\\
	-x, & y>x
\end{cases}$$
Then
$$\int_0^1 v'(y)G_y(x,y)dy = \int_0^x v'(y)G_y(x,y)dy + \int_x^1 v'(y)G_y(x,y)dy$$
$$= (1-x)\int_0^x v'(y)dy - x\int_x^1 v'(y)dy
= \int_0^x v'(y)dy - x\int_0^x v'(y)dy - x\int_x^1 v'(y)dy
= \int_0^x v'(y)dy - x\int_0^1 v'(y)dy$$
$$= v(x) - v(0) - x[v(1) - v(0)]
= v(x) - 0 - x[0 - 0]
= v(x)$$
Since $u\in H^1_0([0,1])$, the identity in particular holds for $u$.


\item The linear interpolant $I_hu$, as a linear combination of the basis functions $\vp_j$, is
$$I_hu = \sum_{j=1}^n u(x_j)\vp_j$$


\item \pf Note that the function $\psi_j$ is not differentiable at $x_j$, but is differentiable at all other points.

Suppose the functions $\psi_j$ are linearly dependent, i.e. there exist scalars $c_j$, not all 0, so that
$$\sum_{j=1}^nc_j\psi_j=0$$
In particular the linear combination
$$\sum_{j=1}^nc_j\psi_j$$
is differentiable. From the assumption we can pick $i$ so that $c_i\ne0$. Since the functions $\psi_j$, for $j\ne i$, are differentiable at $x_i$, the linear combination
$$\sum_{j=1 \atop j\ne i}^n c_j\psi_j$$
is differentiable at $x_i$. Also, since $\psi_i$ is not differentiable at $x_i$ and $c_i\ne0$, the function $c_i\psi_i$ is not differentiable at $x_i$. This means the linear combination $\sum_{j=1}^nc_j\psi_j$ is not differentiable at $x_i$, a contradiction. Thus the functions $\psi_j$ are linearly independent.

Lastly, since $\dim(S_h)=n$, the $n$ linearly independent functions $\psi_j$ form a basis of $S_h$.


\item \pf Recall that
$$G_y(x,y) =
\begin{cases}
	1-x, & y<x\\
	-x, & y>x
\end{cases},
\quad \vp_i'(x) =
\begin{cases}
	0, & x<x_{i-1} \text{ or } x>x_{i+1}\\
	\frac{1}{x_i-x_{i-1}}, & x_{i-1}<x<x_i\\
	-\frac{1}{x_{i+1}-x_i}, & x_i<x<x_{i+1}
\end{cases}$$
First write
$$\int_0^1 [I_hu]'(y)G_y(x_j,y)dy
= \int_0^1 \sbr{\sum_{i=1}^n u(x_i)\vp_i'(y)}G_y(x_j,y)dy$$
$$= \sum_{i=1}^n u(x_i)\int_0^1 \vp_i'(y)G_y(x_j,y)dy
= \sum_{i=1}^n u(x_i)\underbrace{\int_{x_{i-1}}^{x_{i+1}} \vp_i'(y)G_y(x_j,y)dy}_{=:a_{ij}}$$

Now examine cases.
\begin{itemize}
	
	\item If $i=j$,
	$$a_{ij} = \int_{x_{j-1}}^{x_j}\frac{1-x_j}{x_j-x_{j-1}}dy + \int_{x_j}^{x_{j+1}}\frac{-x_j}{-(x_{j+1}-x_j)}dy
	= 1 - x_j + x_j
	= 1$$
	
	\item If $i\ge j+1$,
	$$a_{ij} = \int_{x_{i-1}}^{x_i}\frac{-x_j}{x_i-x_{i-1}}dy + \int_{x_i}^{x_{i+1}}\frac{-x_j}{-(x_{i+1}-x_i)}dy
	= -x_j + x_j
	= 0$$
	
	\item If $i\le j-1$,
	$$a_{ij} = \int_{x_{i-1}}^{x_i}\frac{1-x_j}{x_i-x_{i-1}}dy + \int_{x_i}^{x_{i+1}}\frac{1-x_j}{-(x_{i+1}-x_i)}dy
	= 1 - x_j - (1 - x_j)
	= 0$$	
	
\end{itemize}
This means $a_{ij}=\delta_{ij}$, thus
$$\int_0^1 [I_hu]'(y)G_y(x_j,y)dy = \sum_{i=1}^n u(x_i)\delta_{ij} = u(x_j)$$


\item \pf e
	
\end{enumerate}



\tbf{Problem 3.}

\begin{enumerate}[label=(\alph*)]
	
\item \pf Using the identity $\div(fv)=\grad f\cdot v+f\div v$ for any scalar function $f$ and vector field $v$,
$$\div(e^{-\beta V(x)}\grad u) + \beta e^{-\beta V(x)}f(x)\cdot\grad u
= e^{-\beta V(x)}\Delta u - \beta e^{-\beta V(x)}\grad V(x)\cdot\grad u + \beta e^{-\beta V(x)}f(x)\cdot\grad u$$
$$= \beta e^{-\beta V(x)}\sbr{\beta\inv\Delta u - \grad V(x)\cdot\grad u + f(x)\cdot\grad u}
= \beta e^{-\beta V(x)}\sbr{\beta\inv\Delta u + b(x)\cdot\grad u}$$
and since $\beta e^{-\beta V(x)}\ne0$ for all $x$,
$$\beta\inv\Delta u + b(x)\cdot\grad u = 0
\iff \div(e^{-\beta V(x)}\grad u) + \beta e^{-\beta V(x)}f(x)\cdot\grad u = 0$$
hence the two BVPs are equivalent.


\item Decompose
$$b(x,y) = \m{x-x^3-10xy^2 \\ -y-x^2y}
= \underbrace{\m{x-x^3-xy^2 \\ -y-3y^3-x^2y}}_{=:F(x,y)} + \underbrace{\m{-9xy^2 \\ 3y^3}}_{=:f(x,y)}$$
We check
$$\curl F = \ptl_x(-y-3y^3-x^2y) + \ptl_y(x-x^3-xy^2) = -2xy + 2xy = 0$$
$$\div f = \ptl_x(-9xy^2) + \ptl_y(3y^3) = -9y^2 + 9y^2 = 0$$
so the decomposition is as desired. For the vector field
$$-F(x,y) = \m{-x+x^3+xy^2 \\ y+3y^3+x^2y}$$
a potential $V$ is found by
$$V(x,y) = \int (-x+x^3+xy^2)dx = -\frac12x^2 + \frac14x^4 + \frac12x^2y^2 + g(y)$$
for some function $g$, and
$$V(x,y) = \int (y+3y^3+x^2y)dy = \frac12y^2 + \frac34y^4 + \frac12x^2y^2 + h(x)$$
for some function $h$. Putting the calculations together,
$$V(x,y) = -\frac12x^2 + \frac14x^4 + \frac12x^2y^2 + \frac12y^2 + \frac34y^4$$

 
\end{enumerate}

	
\end{document}