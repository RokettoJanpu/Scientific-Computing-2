\documentclass{article}
\usepackage{amsfonts, amsthm, amsmath, amssymb, mathtools, ulem, mathrsfs, physics, esint, siunitx, tikz-cd}
\usepackage{pdfpages, fullpage, color, microtype, cancel, textcomp, markdown, hyperref, graphicx}
\usepackage{enumitem}
\usepackage{algorithm}
\usepackage{algpseudocode}
\graphicspath{{./images/}}
\usepackage[english]{babel}
\usepackage[autostyle, english=american]{csquotes}
\MakeOuterQuote{"}
\usepackage{xparse}
\usepackage{tikz}

\usepackage{calligra}
\DeclareMathAlphabet{\mathcalligra}{T1}{calligra}{m}{n}
\DeclareFontShape{T1}{calligra}{m}{n}{<->s*[2.2]callig15}{}
\newcommand{\script}[1]{\ensuremath{\mathcalligra{#1}}}
\newcommand{\scr}{\script r}

% fonts
\def\mbb#1{\mathbb{#1}}
\def\mfk#1{\mathfrak{#1}}
\def\mbf#1{\mathbf{#1}}
\def\tbf#1{\textbf{#1}}

% common bold letters
\def\bP{\mbb{P}}
\def\bC{\mbb{C}}
\def\bH{\mbb{H}}
\def\bI{\mbb{I}}
\def\bR{\mbb{R}}
\def\bQ{\mbb{Q}}
\def\bZ{\mbb{Z}}
\def\bN{\mbb{N}}

% brackets
\newcommand{\br}[1]{\left(#1\right)}
\newcommand{\sbr}[1]{\left[#1\right]}
\newcommand{\brc}[1]{\left\{#1\right\}}
\newcommand{\lbr}[1]{\left\langle#1\right\rangle}

% vectors
\renewcommand{\i}{\hat{\imath}}
\renewcommand{\j}{\hat{\jmath}}
\renewcommand{\k}{\hat{k}}
\newcommand{\proj}[2]{\text{proj}_{#2}\br{#1}}
\newcommand{\m}[2][b]{\begin{#1matrix}#2\end{#1matrix}}
\newcommand{\arr}[3][\sbr]{#1{\begin{array}{#2}#3\end{array}}}

% misc
\NewDocumentCommand{\seq}{O{n} O{1} O{\infty} m}{\br{#4}_{{#1}={#2}}^{#3}}
\NewDocumentCommand{\app}{O{x} O{\infty}}{\xrightarrow{#1\to#2}}
\newcommand{\sm}{\setminus}
\newcommand{\sse}{\subseteq}
\renewcommand{\ss}{\subset}
\newcommand{\vn}{\varnothing}
\newcommand{\lc}{\epsilon_{ijk}}
\newcommand{\ep}{\epsilon}
\newcommand{\vp}{\varphi}
\renewcommand{\th}{\theta}
\newcommand{\cjg}[1]{\overline{#1}}
\newcommand{\inv}{^{-1}}
\DeclareMathOperator{\im}{im}
\DeclareMathOperator{\id}{id}
\newcommand{\ans}{\tbf{Ans. }}
\newcommand{\pf}{\tbf{Pf. }}
\newcommand{\imp}{\implies}
\newcommand{\impleft}{\reflectbox{$\implies$}}
\newcommand{\ck}{\frac1{4\pi\ep_0}}
\newcommand{\ckb}{4\pi\ep_0}
\newcommand{\sto}{\longrightarrow}
\DeclareMathOperator{\cl}{cl}
\DeclareMathOperator{\intt}{int}
\DeclareMathOperator{\bd}{bd}
\DeclareMathOperator{\Span}{span}
\newcommand{\floor}[1]{\left\lfloor#1\right\rfloor}
\newcommand{\ceil}[1]{\left\lceil#1\right\rceil}
\newcommand{\fxn}[5]{#1:\begin{array}{rcl}#2&\longrightarrow & #3\\[-0.5mm]#4&\longmapsto &#5\end{array}}
\newcommand{\sep}[1][.5cm]{\vspace{#1}}
\DeclareMathOperator{\card}{card}
\renewcommand{\ip}[2]{\lbr{#1,#2}}
\renewcommand{\bar}{\overline}
\DeclareMathOperator{\cis}{cis}
\DeclareMathOperator{\Arg}{Arg}
\newcommand{\ptl}{\partial}
\newcommand{\om}{\omega}

% title
\title{Scientific Computing HW 5}
\author{Ryan Chen}
%\date{\today}
\setlength{\parindent}{0pt}


\begin{document}
	
\maketitle



\tbf{Problem 1.}

\begin{enumerate}
	
\item From $H(p,q)=T(p)=U(q)$,
$$\ptl_pH(p,q) = T'(p),
\quad \ptl_qH(p,q) = U'(q)$$
Plug into the Stoermer--Verlet method.
$$p_{n+1/2} = p_n - \frac12hU'(q_n)$$
$$q_{n+1} = q_n + \frac12h[T'(p_{n+1/2}) + T'(p_{n+1/2})]
= q_n + hT'\br{p_n - \frac12hU'(q_n)}$$
$$p_{n+1} = p_n - \frac12hU'(q_n) - \frac12hU'(q_{n+1})
= p_n - \frac12h\sbr{U'(q_n) + U'\br{q_n + hT'\br{p_n - \frac12hU'(q_n)}}}$$
The RHS quantities are independent of $p_{n+1},q_{n+1}$, so the method is explicit.\\

The Hamiltonian for the 1D simple harmonic oscillator is
$$H(p,q) = T(p) + U(q),
\quad T(p) := \frac{p^2}{2m},
\quad U(q) := \frac{m\om^2q^2}{2}$$
First compute
$$T'(p) = \frac pm,
\quad U'(q) = m\om^2q$$
Plug into the method.
$$q_{n+1} = q_n + hT'\br{p_n - \frac12hm\om^2q_n}
= q_n + \frac hm\sbr{p_n - \frac12h\om^2q_n}
= \frac hmp_n + \br{1 - \frac12h^2\om^2}q_n$$
$$p_{n+1} = p_n - \frac12h\sbr{m\om^2q_n + m\om^2\br{q_n + \frac hm\br{p_n - \frac12hm\om^2q_n}}}$$
In the above expression, collect coefficients of the following terms.
\begin{align*}
	p_n &: \quad 1 - \frac12hm\om^2\frac hm = 1 - \frac12h^2\om^2 \\
	q_n &: \quad -\frac12h\sbr{m\om^2 + m\om^2\br{1+ \frac hm\br{-\frac12hm\om^2}}}
	= -\frac12hm\om^2\br{2 - \frac12h^2\om^2}
	= hm\om^2\br{\frac14h^2\om^2 - 1}
\end{align*}
Therefore
$$\m{p_{n+1} \\ q_{n+1}} = A\m{p_n \\ q_n},
\quad A := \m{a & b \\ c & a},
\quad a := 1 - \frac12h^2\om^2,
\quad b := hm\om^2\br{\frac14h^2\om^2 - 1},
\quad c := \frac hm$$


\item We compute
$$JA = \m{0 & 1 \\ -1 & 0}\m{a & b \\ c & a}
= \m{c & a \\ -a & -b}$$
$$\imp A^TJA = \m{a & c \\ b & a}\m{c & a \\ -a & -b}
= \m{ac-ca & a^2-bc \\ bc-a^2 & ba-ab}
= \m{0 & a^2-bc \\ -(a^2-bc) & 0}$$
$$a^2 - bc = 1 + \frac14h^4\om^4 - h^2\om^2 - h^2\om^2\br{\frac14h^2\om^2 - 1}
= 1 + \frac14h^4\om^4 - h^2\om^2 - \frac14h^4\om^4 + h^2\om^2
= 1$$
$$\imp A^TJA = \m{0 & 1 \\ -1 & 0} = J$$


\item e

\end{enumerate}

	
\end{document}