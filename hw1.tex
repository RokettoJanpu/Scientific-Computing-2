\documentclass{article}
\usepackage{amsfonts, amsthm, amsmath, amssymb, mathtools, ulem, mathrsfs, physics, esint, siunitx, tikz-cd}
\usepackage{pdfpages, fullpage, color, microtype, cancel, textcomp, markdown, hyperref, graphicx}
\usepackage{enumitem}
\usepackage{algorithm}
\usepackage{algpseudocode}
\graphicspath{{./images/}}
\usepackage[english]{babel}
\usepackage[autostyle, english=american]{csquotes}
\MakeOuterQuote{"}
\usepackage{xparse}
\usepackage{tikz}

\usepackage{calligra}
\DeclareMathAlphabet{\mathcalligra}{T1}{calligra}{m}{n}
\DeclareFontShape{T1}{calligra}{m}{n}{<->s*[2.2]callig15}{}
\newcommand{\script}[1]{\ensuremath{\mathcalligra{#1}}}
\newcommand{\scr}{\script r}

% fonts
\def\mbb#1{\mathbb{#1}}
\def\mfk#1{\mathfrak{#1}}
\def\mbf#1{\mathbf{#1}}
\def\tbf#1{\textbf{#1}}

% common bold letters
\def\bP{\mbb{P}}
\def\bC{\mbb{C}}
\def\bH{\mbb{H}}
\def\bI{\mbb{I}}
\def\bR{\mbb{R}}
\def\bQ{\mbb{Q}}
\def\bZ{\mbb{Z}}
\def\bN{\mbb{N}}

% brackets
\newcommand{\br}[1]{\left(#1\right)}
\newcommand{\sbr}[1]{\left[#1\right]}
\newcommand{\brc}[1]{\left\{#1\right\}}
\newcommand{\lbr}[1]{\left\langle#1\right\rangle}

% vectors
\renewcommand{\i}{\hat{\imath}}
\renewcommand{\j}{\hat{\jmath}}
\renewcommand{\k}{\hat{k}}
\newcommand{\proj}[2]{\text{proj}_{#2}\br{#1}}
\newcommand{\m}[2][b]{\begin{#1matrix}#2\end{#1matrix}}
\newcommand{\arr}[3][\sbr]{#1{\begin{array}{#2}#3\end{array}}}

% misc
\NewDocumentCommand{\seq}{O{n} O{1} O{\infty} m}{\br{#4}_{{#1}={#2}}^{#3}}
\NewDocumentCommand{\app}{O{x} O{\infty}}{\xrightarrow{#1\to#2}}
\newcommand{\sm}{\setminus}
\newcommand{\sse}{\subseteq}
\renewcommand{\ss}{\subset}
\newcommand{\vn}{\varnothing}
\newcommand{\lc}{\epsilon_{ijk}}
\newcommand{\ep}{\epsilon}
\newcommand{\vp}{\varphi}
\renewcommand{\th}{\theta}
\newcommand{\cjg}[1]{\overline{#1}}
\newcommand{\inv}{^{-1}}
\DeclareMathOperator{\im}{im}
\DeclareMathOperator{\id}{id}
\newcommand{\ans}{\tbf{Ans. }}
\newcommand{\pf}{\tbf{Pf. }}
\newcommand{\imp}{\implies}
\newcommand{\impleft}{\reflectbox{$\implies$}}
\newcommand{\ck}{\frac1{4\pi\ep_0}}
\newcommand{\ckb}{4\pi\ep_0}
\newcommand{\sto}{\longrightarrow}
\DeclareMathOperator{\cl}{cl}
\DeclareMathOperator{\intt}{int}
\DeclareMathOperator{\bd}{bd}
\DeclareMathOperator{\Span}{span}
\newcommand{\floor}[1]{\left\lfloor#1\right\rfloor}
\newcommand{\ceil}[1]{\left\lceil#1\right\rceil}
\newcommand{\fxn}[5]{#1:\begin{array}{rcl}#2&\longrightarrow & #3\\[-0.5mm]#4&\longmapsto &#5\end{array}}
\newcommand{\sep}[1][.5cm]{\vspace{#1}}
\DeclareMathOperator{\card}{card}
\renewcommand{\ip}[2]{\lbr{#1,#2}}
\renewcommand{\bar}{\overline}
\DeclareMathOperator{\cis}{cis}
\DeclareMathOperator{\Arg}{Arg}

% title
\title{Scientific Computing HW 1}
\author{Ryan Chen}
%\date{\today}
\setlength{\parindent}{0pt}


\begin{document}

\maketitle



\tbf{P1.} Pick $T<t^*$ where $t^*:=t_0+\frac{1}{y_0}$. Fix $r\in\bR$. Then $f(t,y):=y^2$ is continuous on the cylinder $Q:=\brc{t_0\le t\le T,~|y-y_0|\le r}$. Now for all $(t,y)\in Q$,
$$|y-y_0|\le r
\imp |y| = |y-y_0+y_0|
\le |y-y_0|+|y_0|
\le r+|y_0|
\imp |f(t,y)| = |y^2| = |y|^2 \le (r+|y_0|)^2$$
i.e. $|f|\le M$ on $Q$ where $M:=(r+|y_0|)^2$. Then by theorem 1, the IVP has a solution for $0\le t-t_0\le \min(\tfrac rM,T-t_0)$.
\sep



\tbf{P2.} Pick $0\le T<\infty$. Fix $r\in\bR$. Then $f(t,y):=2y^{1/2}$ is continuous on the cylinder $Q:=\brc{t_0\le t\le T,~|y-y_0|\le r}$. Now for all $(t,y)\in Q$ with $y\ge0$, using similar arguments as in P1,
$$|y-y_0|\le r
\imp y = |y| \le r+|y_0|
\imp |f(t,y)| = |2y^{1/2}| = 2y^{1/2} \le 2(r+|y_0|)^{1/2}$$
i.e. $|f|\le M$ on $Q$ where $M:=2(r+|y_0|)^{1/2}$. Then by theorem 1, the IVP has a solution for $0\le t-t_0\le \min(\frac rM,T-t_0)$.\\

To see that $f(y):=2y^{1/2}$ is not Lipschitz at $y=0$, suppose there exists $C>0$ such that $|f(x)-f(0)|\le C|x-0|$ for all $x\ge0$, i.e. $2x^{1/2}\le Cx$. But if we set $x=\frac{1}{C^2}$, we get $2\le Cx^{1/2}=C\frac1C=1$, a contradiction.
\sep



\tbf{P3.} We compute the Picard iterates for $f(t,y):=y^2,~t_0=0,~y_0=1$, starting at $y_1(t)=1$.
$$y_2(t) = 1 + \int_0^t f(s,y_1(s))ds = 1 + \int_0^t 1ds = 1 + t$$
$$y_3(t) = 1 + \int_0^t f(s,y_2(s))ds = 1 + \int_0^t (1+s)^2ds = 1 + \int_0^t (1+2s+s^2)ds = 1+t+t^2+\frac13t^3$$
$$y_4(t) = 1 + \int_0^t\br{1+2s+3s^2+\frac83s^3+\frac53s^4+\frac23s^5+\frac19s^6}ds = 1+t+t^2+t^3+\frac23t^4+\frac13t^5+\frac19t^6+\frac1{63}t^7$$
The Picard iterates seem to converge to $\frac1{1-t}$ for $t<1$. Plugging values into the time interval given in P1, we get
$$M = (r+|y_0|)^2 = (r+1)^2$$
$$\frac rM = \frac{r}{(r+1)^2} \le 1$$
$$T-t_0 = T < t^* = t_0+\frac{1}{y_0} = 1$$
$$0 \le t \le \min\br{\frac rM,T-t_0} < 1$$
which is a smaller interval.
\sep



\tbf{P4.} \pf Using the Taylor expansion of $y$ at $t_n$,
\begin{align*}
	\tau_{n+1} &= y + hy' + \frac12h^2y'' + \frac16h^3y''' + O(h^4) - y \\
	& - h\sbr{\frac{23}{12}y' - \frac43\br{y' - hy'' + \frac12h^2y''' - \frac16h^3y'''' + O(h^4)} + \frac{5}{12}\br{y' - 2hy'' + 2h^2y''' - \frac43h^3y'''' + O(h^4)}}
\end{align*}
Collect coefficients of the following terms:
$$y: \quad 1 - 1 = 0$$
$$hy': \quad 1 - \frac{23}{12} + \frac43 - \frac{5}{12} = \frac{1}{12}(12-23+16-5) = 0$$
$$h^2y'': \quad \frac12 - \frac43 + \frac56 = \frac16(3-8+5) = 0$$
$$h^3y''': \quad \frac16 + \frac23 - \frac56 = \frac16(1+4-5) = 0$$
$$h^4y'''': \quad \frac{1}{24} - \frac29 + \frac59 \ne 0$$
This gives $\tau_{n+1}=O(h^4)$. Thus the method is consistent of order 3.
\sep



\tbf{P5.} 



\tbf{P6.}

\begin{enumerate}[label=(\alph*)]
	
\item Since we have four undetermined coefficients, we take a third order Taylor expansion. Any higher order expansion would give an overconstrained system.
\begin{align*}
	\tau_{n+1} &= y + hy' + \frac12h^2y'' + \frac16h^3y''' + O(h^4) - a_0y - a_1\sbr{y' - hy' + \frac12h^2y'' - \frac16h^3y''' + O(h^4)}\\
	& - h\sbr{b_0y' + b_1(y' - hy'' + \frac12h^2y''' - \frac16h^3y'''' + O(h^4))}
\end{align*}
Collect coefficients of the following terms. To seek consistency of the highest possible order (in this case order 3), set each sum equal to 0:
$$y: \quad 1 - a_0 - a_1 = 0$$
$$hy': \quad 1 + a_1 - b_0 - b_1 = 0$$
$$h^2y'': \quad \frac12 - \frac12a_1 + b_1 = 0$$
$$h^3y''': \quad \frac16+ \frac16a_1 - \frac12b_1 = 0$$
These equations form a linear system whose solution is $a_0=-4,~a_1=5,~b_0=4,~b_1=2$.


\item Applying the method to $y'=0=:f(t,y)$ with initial condition $y(0)=a$,
$$u_{n+1} + 4u_n - 5u_{n-1} = 0$$
Using the ansatz solution $r^n$,
$$0 = r^2+4r-5 = (r+5)(r-1)
\imp r=1,-5
\imp u_n = A + B(-5)^n$$
The exact solution of the IVP is $y(t)=a$. Perturb the values of the first two iterates, say $u_0=a+\delta_0$ and $u_1=a+\delta_1$. Then
$$a+\delta_0 = A+B,~a+\delta_1 = A-5B
\imp \delta_0 - \delta_1 = 6B$$
If $\delta_0\ne\delta_1$ then $B\ne0$, in which case the solution blows up. Thus the method is unstable.

\item LINK

\end{enumerate}

\end{document}